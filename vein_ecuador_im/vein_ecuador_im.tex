%  LaTeX support: latex@mdpi.com
%  For support, please attach all files needed for compiling as well as the log file, and specify your operating system, LaTeX version, and LaTeX editor.

%=================================================================
% pandoc conditionals added to preserve backwards compatibility with previous versions of rticles

\documentclass[atmosphere,article,submit,moreauthors,pdftex]{Definitions/mdpi}


%% Some pieces required from the pandoc template
\setlist[itemize]{leftmargin=*,labelsep=5.8mm}
\setlist[enumerate]{leftmargin=*,labelsep=4.9mm}


%--------------------
% Class Options:
%--------------------

%---------
% article
%---------
% The default type of manuscript is "article", but can be replaced by:
% abstract, addendum, article, book, bookreview, briefreport, casereport, comment, commentary, communication, conferenceproceedings, correction, conferencereport, entry, expressionofconcern, extendedabstract, datadescriptor, editorial, essay, erratum, hypothesis, interestingimage, obituary, opinion, projectreport, reply, retraction, review, perspective, protocol, shortnote, studyprotocol, systematicreview, supfile, technicalnote, viewpoint, guidelines, registeredreport, tutorial
% supfile = supplementary materials

%----------
% submit
%----------
% The class option "submit" will be changed to "accept" by the Editorial Office when the paper is accepted. This will only make changes to the frontpage (e.g., the logo of the journal will get visible), the headings, and the copyright information. Also, line numbering will be removed. Journal info and pagination for accepted papers will also be assigned by the Editorial Office.

%------------------
% moreauthors
%------------------
% If there is only one author the class option oneauthor should be used. Otherwise use the class option moreauthors.

%---------
% pdftex
%---------
% The option pdftex is for use with pdfLaTeX. Remove "pdftex" for (1) compiling with LaTeX & dvi2pdf (if eps figures are used) or for (2) compiling with XeLaTeX.

%=================================================================
% MDPI internal commands - do not modify
\firstpage{1}
\makeatletter
\setcounter{page}{\@firstpage}
\makeatother
\pubvolume{1}
\issuenum{1}
\articlenumber{0}
\pubyear{2023}
\copyrightyear{2023}
%\externaleditor{Academic Editor: Firstname Lastname}
\datereceived{ }
\daterevised{ } % Comment out if no revised date
\dateaccepted{ }
\datepublished{ }
%\datecorrected{} % For corrected papers: "Corrected: XXX" date in the original paper.
%\dateretracted{} % For corrected papers: "Retracted: XXX" date in the original paper.
\hreflink{https://doi.org/} % If needed use \linebreak
%\doinum{}
%\pdfoutput=1 % Uncommented for upload to arXiv.org

%=================================================================
% Add packages and commands here. The following packages are loaded in our class file: fontenc, inputenc, calc, indentfirst, fancyhdr, graphicx, epstopdf, lastpage, ifthen, float, amsmath, amssymb, lineno, setspace, enumitem, mathpazo, booktabs, titlesec, etoolbox, tabto, xcolor, colortbl, soul, multirow, microtype, tikz, totcount, changepage, attrib, upgreek, array, tabularx, pbox, ragged2e, tocloft, marginnote, marginfix, enotez, amsthm, natbib, hyperref, cleveref, scrextend, url, geometry, newfloat, caption, draftwatermark, seqsplit
% cleveref: load \crefname definitions after \begin{document}

%=================================================================
% Please use the following mathematics environments: Theorem, Lemma, Corollary, Proposition, Characterization, Property, Problem, Example, ExamplesandDefinitions, Hypothesis, Remark, Definition, Notation, Assumption
%% For proofs, please use the proof environment (the amsthm package is loaded by the MDPI class).

%=================================================================
% Full title of the paper (Capitalized)
\Title{Quantifying the Impact of High Emitters on Vehicle CO Emissions
An Analysis of Ecuador's Inspection and Maintenance Program}

% MDPI internal command: Title for citation in the left column
\TitleCitation{Quantifying the Impact of High Emitters on Vehicle CO
Emissions An Analysis of Ecuador's Inspection and Maintenance Program}

% Author Orchid ID: enter ID or remove command
%\newcommand{\orcidauthorA}{0000-0000-0000-000X} % Add \orcidA{} behind the author's name
%\newcommand{\orcidauthorB}{0000-0000-0000-000X} % Add \orcidB{} behind the author's name


% Authors, for the paper (add full first names)
\Author{Sergio
Ibarra-Espinosa$^{1,2*}$\href{https://orcid.org/0000-0002-3162-1905}
{\orcidicon}, Zamir
Mera$^{3,4}$\href{https://orcid.org/0000-0003-2897-8847}
{\orcidicon}, Karl
Ropkins$^{5}$\href{https://orcid.org/0000-0002-0294-6997}
{\orcidicon}}


%\longauthorlist{yes}


% MDPI internal command: Authors, for metadata in PDF
\AuthorNames{Sergio Ibarra-Espinosa, Zamir Mera, Karl Ropkins}

% MDPI internal command: Authors, for citation in the left column

% Affiliations / Addresses (Add [1] after \address if there is only one affiliation.)
\address{%
$^{1}$ \quad Cooperative Institute for Research in Environmental
Sciences, University of Colorado-Boulder, Boulder, CO, United
States; \href{mailto:sergio.ibarraespinosa@colorado.edu}{\nolinkurl{sergio.ibarraespinosa@colorado.edu}}\\
$^{2}$ \quad NOAA Global Monitoring Laboratory Boulder, CO, United
States; \\
$^{3}$ \quad Faculty of Applied Sciences, Universidad Técnica del Norte,
Ibarra, Ecuador; \\
$^{4}$ \quad Fundación Alma Verde, Ibarra, Ecuador; \\
$^{5}$ \quad Institute for Transport Studies, University of Leeds,
United Kingdom; \\
}

% Contact information of the corresponding author
\corres{Correspondence: \href{mailto:sergio.ibarraespinosa@colorado.edu}{\nolinkurl{sergio.ibarraespinosa@colorado.edu}}}

% Current address and/or shared authorship








% The commands \thirdnote{} till \eighthnote{} are available for further notes

% Simple summary
\simplesumm{A concise summary of the paper's findings on high emitters
and I/M programs.}

%\conference{} % An extended version of a conference paper

% Abstract (Do not insert blank lines, i.e. \\)
\abstract{Vehicular emissions are a primary contributor to urban air
pollution, and inspection and maintenance (I/M) programs are key
strategies to mitigate this impact. However, the effectiveness of these
programs can be limited by ``high emitters''---vehicles that fail
emission standards but continue to circulate until their next
inspection. This study quantifies and compares vehicular CO emissions in
Quito, Ecuador, under two distinct scenarios using the VEIN (Vehicular
Emissions Inventory) model. The first scenario assumes all vehicles
comply with average emission factors, representing a baseline without
explicit consideration of high emitters. The second scenario explicitly
accounts for the significant contribution of high emitters that pass the
annual inspection and remain in circulation. Our findings highlight the
substantial difference in total emissions between these scenarios,
emphasizing the critical role of undetected high emitters in inflating
overall pollution levels. This understanding can inform the design of
more effective I/M programs and enhance air quality management
strategies in Quito and other similar urban environments.}


% Keywords
\keyword{Vehicular Emissions; Inspection and Maintenance (I/M) Program;
High Emitters; VEIN Model; Air Quality; Ecuador; CO Emissions; Emission
Inventory.}

% The fields PACS, MSC, and JEL may be left empty or commented out if not applicable
%\PACS{J0101}
%\MSC{}
%\JEL{}

%%%%%%%%%%%%%%%%%%%%%%%%%%%%%%%%%%%%%%%%%%
% Only for the journal Diversity
%\LSID{\url{http://}}

%%%%%%%%%%%%%%%%%%%%%%%%%%%%%%%%%%%%%%%%%%
% Only for the journal Applied Sciences

%%%%%%%%%%%%%%%%%%%%%%%%%%%%%%%%%%%%%%%%%%

%%%%%%%%%%%%%%%%%%%%%%%%%%%%%%%%%%%%%%%%%%
% Only for the journal Data



%%%%%%%%%%%%%%%%%%%%%%%%%%%%%%%%%%%%%%%%%%
% Only for the journal Toxins


%%%%%%%%%%%%%%%%%%%%%%%%%%%%%%%%%%%%%%%%%%
% Only for the journal Encyclopedia


%%%%%%%%%%%%%%%%%%%%%%%%%%%%%%%%%%%%%%%%%%
% Only for the journal Advances in Respiratory Medicine
%\addhighlights{yes}
%\renewcommand{\addhighlights}{%

%\noindent This is an obligatory section in “Advances in Respiratory Medicine”, whose goal is to increase the discoverability and readability of the article via search engines and other scholars. Highlights should not be a copy of the abstract, but a simple text allowing the reader to quickly and simplified find out what the article is about and what can be cited from it. Each of these parts should be devoted up to 2~bullet points.\vspace{3pt}\\
%\textbf{What are the main findings?}
% \begin{itemize}[labelsep=2.5mm,topsep=-3pt]
% \item First bullet.
% \item Second bullet.
% \end{itemize}\vspace{3pt}
%\textbf{What is the implication of the main finding?}
% \begin{itemize}[labelsep=2.5mm,topsep=-3pt]
% \item First bullet.
% \item Second bullet.
% \end{itemize}
%}


%%%%%%%%%%%%%%%%%%%%%%%%%%%%%%%%%%%%%%%%%%

% Pandoc syntax highlighting
\usepackage{color}
\usepackage{fancyvrb}
\newcommand{\VerbBar}{|}
\newcommand{\VERB}{\Verb[commandchars=\\\{\}]}
\DefineVerbatimEnvironment{Highlighting}{Verbatim}{commandchars=\\\{\}}
% Add ',fontsize=\small' for more characters per line
\usepackage{framed}
\definecolor{shadecolor}{RGB}{248,248,248}
\newenvironment{Shaded}{\begin{snugshade}}{\end{snugshade}}
\newcommand{\AlertTok}[1]{\textcolor[rgb]{0.94,0.16,0.16}{#1}}
\newcommand{\AnnotationTok}[1]{\textcolor[rgb]{0.56,0.35,0.01}{\textbf{\textit{#1}}}}
\newcommand{\AttributeTok}[1]{\textcolor[rgb]{0.13,0.29,0.53}{#1}}
\newcommand{\BaseNTok}[1]{\textcolor[rgb]{0.00,0.00,0.81}{#1}}
\newcommand{\BuiltInTok}[1]{#1}
\newcommand{\CharTok}[1]{\textcolor[rgb]{0.31,0.60,0.02}{#1}}
\newcommand{\CommentTok}[1]{\textcolor[rgb]{0.56,0.35,0.01}{\textit{#1}}}
\newcommand{\CommentVarTok}[1]{\textcolor[rgb]{0.56,0.35,0.01}{\textbf{\textit{#1}}}}
\newcommand{\ConstantTok}[1]{\textcolor[rgb]{0.56,0.35,0.01}{#1}}
\newcommand{\ControlFlowTok}[1]{\textcolor[rgb]{0.13,0.29,0.53}{\textbf{#1}}}
\newcommand{\DataTypeTok}[1]{\textcolor[rgb]{0.13,0.29,0.53}{#1}}
\newcommand{\DecValTok}[1]{\textcolor[rgb]{0.00,0.00,0.81}{#1}}
\newcommand{\DocumentationTok}[1]{\textcolor[rgb]{0.56,0.35,0.01}{\textbf{\textit{#1}}}}
\newcommand{\ErrorTok}[1]{\textcolor[rgb]{0.64,0.00,0.00}{\textbf{#1}}}
\newcommand{\ExtensionTok}[1]{#1}
\newcommand{\FloatTok}[1]{\textcolor[rgb]{0.00,0.00,0.81}{#1}}
\newcommand{\FunctionTok}[1]{\textcolor[rgb]{0.13,0.29,0.53}{\textbf{#1}}}
\newcommand{\ImportTok}[1]{#1}
\newcommand{\InformationTok}[1]{\textcolor[rgb]{0.56,0.35,0.01}{\textbf{\textit{#1}}}}
\newcommand{\KeywordTok}[1]{\textcolor[rgb]{0.13,0.29,0.53}{\textbf{#1}}}
\newcommand{\NormalTok}[1]{#1}
\newcommand{\OperatorTok}[1]{\textcolor[rgb]{0.81,0.36,0.00}{\textbf{#1}}}
\newcommand{\OtherTok}[1]{\textcolor[rgb]{0.56,0.35,0.01}{#1}}
\newcommand{\PreprocessorTok}[1]{\textcolor[rgb]{0.56,0.35,0.01}{\textit{#1}}}
\newcommand{\RegionMarkerTok}[1]{#1}
\newcommand{\SpecialCharTok}[1]{\textcolor[rgb]{0.81,0.36,0.00}{\textbf{#1}}}
\newcommand{\SpecialStringTok}[1]{\textcolor[rgb]{0.31,0.60,0.02}{#1}}
\newcommand{\StringTok}[1]{\textcolor[rgb]{0.31,0.60,0.02}{#1}}
\newcommand{\VariableTok}[1]{\textcolor[rgb]{0.00,0.00,0.00}{#1}}
\newcommand{\VerbatimStringTok}[1]{\textcolor[rgb]{0.31,0.60,0.02}{#1}}
\newcommand{\WarningTok}[1]{\textcolor[rgb]{0.56,0.35,0.01}{\textbf{\textit{#1}}}}

% tightlist command for lists without linebreak
\providecommand{\tightlist}{%
  \setlength{\itemsep}{0pt}\setlength{\parskip}{0pt}}



\usepackage{longtable}

\begin{document}



%%%%%%%%%%%%%%%%%%%%%%%%%%%%%%%%%%%%%%%%%%

\section{Introduction}\label{introduction}

ntroduction: * Background: Discuss the significance of vehicular
emissions in urban air quality, focusing on pollutants like CO. Mention
the context of Quito, Ecuador, as a developing city with specific air
quality challenges. * Problem Statement: Explain that while I/M programs
are designed to reduce emissions, their effectiveness can be compromised
by vehicles that fail emission tests but remain in use until the next
inspection. These ``high emitters'' can disproportionately contribute to
overall pollution. * Literature Review: * Discuss existing literature on
vehicular emission inventories and modeling, highlighting the role of
the VEIN model (Ibarra-Espinosa et al., 2018). * Review studies on the
effectiveness of I/M programs globally and in Latin America. * Cite
research on the concept and impact of ``high emitters'' or ``gross
polluters.'' * Research Gap: Clearly articulate that a detailed
quantification of the impact of high emitters on total emissions,
specifically within the context of Quito's I/M program, is lacking. *
Objectives: * To estimate CO emissions from the gasoline vehicle fleet
in Quito using the VEIN model. * To quantify the difference in emissions
between a scenario that assumes uniform emission factors (without
explicit high emitters) and a scenario that accounts for high emitters
that pass the annual inspection. * To assess the potential impact of
improving the detection and removal of high emitters within Quito's I/M
program. * Structure of the Paper: Briefly outline the sections of the
manuscript.

\section{Materials and Methods:}\label{materials-and-methods}

\begin{itemize}
\tightlist
\item
  Study Area: Describe Quito, Ecuador, including its geographical
  characteristics, typical traffic patterns, and existing I/M program
  details (e.g., frequency of inspection, detection methods).
\item
  Information about Ecuador's I/M program can be found in sources like
  SGS Ecuador and general government regulations.
\item
  VEIN Model Description:
\item
  Explain the VEIN R package and its suitability for creating
  high-resolution, street-level emission inventories.
\item
  Detail the specific functions used from VEIN, particularly ef\_whe
  (emission factor with high emitters) if applicable, or how you adapted
  the model to incorporate your high-emitter calculation.
\item
  Mention the input data required by VEIN (e.g., vehicle fleet
  composition, age, fuel type, pollutant, emission type, subtype).
\item
  Data Sources:
\item
  Vehicle Fleet Data: Specify how you obtained data on vehicle types
  (e.g., gasoline cars), their distribution, age, and size in Quito.
\item
  Emission Factors:
\item
  Scenario 1 (Without High Emitters): Describe the average emission
  factors used for compliant vehicles (e.g., 0.1 g CO/km as in your
  example). Specify the source of these factors (e.g., standard emission
  databases, local studies).
\item
  Scenario 2 (With High Emitters):
\item
  Explain how you defined ``high emitters'' -- e.g., a multiplier of the
  average emission factor (e.g., 1 g CO/km in your example).
\item
  Detail how you determined the proportion of ``approving'' (normal
  emitters) and ``reproving'' (high emitters) vehicles based on the
  annual inspection data (e.g., 80\% approval, 20\% reproval).
\item
  Explain how these proportions and factors are used to calculate the
  weighted emission factors (as per your example: 0.80 * 0.1 g/km + 0.20
  * 1 g/km = 0.28 g/km).
\item
  Activity Data: Describe how you obtained or estimated vehicle activity
  (e.g., kilometers driven per year, per vehicle type). This is crucial
  for converting emission factors to total emissions.
\item
  Pollutants and Emission Types: Specify which pollutants (CO, as per
  your example) and emission types (e.g., exhaust) are included in the
  analysis.
\item
  Methodology for High Emitter Calculation:
\item
  Provide a detailed explanation of your approach to calculating
  weighted emission factors, using the provided example as a basis.
\item
  Explain how these weighted factors are applied within the VEIN
  framework.
\item
  Describe how you simulate the ``circulation until measurement'' aspect
  -- i.e., that reproving vehicles continue to emit at their higher
  rates until their annual inspection.
\item
  Data Analysis and Modeling:
\item
  Describe how you integrated the calculated emission factors and
  activity data into VEIN to produce emission inventories for both
  scenarios.
\item
  Specify the temporal and spatial resolution of your inventory (e.g.,
  hourly, street-level).
\item
  Comparison Metrics: Define how you will compare the two scenarios
  (e.g., total CO emissions, difference in emissions as a percentage,
  contribution of high emitters to total emissions).
\end{itemize}

\section{Results:}\label{results}

\begin{itemize}
\tightlist
\item
  Vehicle Fleet Characterization: Present a summary of the vehicle fleet
  used in the analysis for Quito (e.g., breakdown by age, type, fuel).
\item
  Emission Factors: Show the calculated weighted emission factors for
  both scenarios (average vs.~high-emitter accounted).
\item
  Total CO Emissions:
\item
  Present the estimated total CO emissions for Quito under Scenario 1
  (without explicit high emitters).
\item
  Present the estimated total CO emissions for Quito under Scenario 2
  (with explicit high emitters).
\item
  Clearly show the difference between the two scenarios.
\item
  Contribution of High Emitters: Quantify the percentage contribution of
  the ``reproving'' (high emitter) vehicles to the total emissions in
  Scenario 2.
\item
  Spatial/Temporal Distribution (Optional but Recommended): If VEIN
  allows, show how emissions differ spatially (e.g., hotspots) or
  temporally between the two scenarios.
\item
  Statistical Significance (if applicable): If you perform any
  statistical tests to compare the scenarios.
\end{itemize}

\section{Discussion:}\label{discussion}

\begin{itemize}
\tightlist
\item
  Interpretation of Findings:
\item
  Discuss the magnitude of the difference in emissions between the two
  scenarios. Was it as expected?
\item
  Explain why high emitters have such a significant impact, linking it
  to their higher emission rates.
\item
  Relate your findings to the effectiveness of Quito's current I/M
  program. Does the annual inspection adequately capture and remove high
  emitters?
\item
  Implications for I/M Programs:
\item
  Discuss how the findings inform policy decisions related to I/M
  programs, such as the frequency of testing, the stringency of test
  limits, and the enforcement mechanisms.
\item
  Consider the potential benefits of more frequent or more sensitive
  emission testing.
\item
  Discuss the economic and environmental trade-offs.
\item
  VEIN Model Application: Comment on the utility of the VEIN model for
  this type of analysis, particularly its ability to handle detailed
  emissions calculations.
\item
  Limitations:
\item
  Acknowledge any limitations in the data (e.g., accuracy of fleet data,
  activity data, emission factors).
\item
  Discuss the assumptions made (e.g., constant emission rates for high
  emitters until inspection, homogeneity of vehicle population within
  categories).
\item
  Mention any simplifications in the I/M program representation.
\item
  Comparison with Previous Studies: Place your findings in the context
  of existing research on vehicular emissions and I/M programs.
\item
  Future Research: Suggest areas for further investigation (e.g.,
  including other pollutants, different vehicle types, other cities, the
  impact of different I/M strategies).
\end{itemize}

\section{Conclusions:}\label{conclusions}

\begin{itemize}
\tightlist
\item
  Summarize the main findings regarding the impact of high emitters on
  CO emissions in Quito.
\item
  Reiterate the importance of effectively identifying and addressing
  high emitters in I/M programs.
\item
  Provide a concise policy recommendation for improving air quality
  management in Quito based on your results.
\end{itemize}

This Rmd-skeleton uses the mdpi Latex template published 2023-03-25.
However, the official template gets more frequently updated than the
\textbf{rticles} package. Therefore, please make sure prior to paper
submission, that you're using the most recent .cls, .tex and .bst files
(available \href{http://www.mdpi.com/authors/latex}{here}).

\section{Article Header Information}\label{article-header-information}

The YAML header includes information needed mainly for formatting the
front and back matter of the article. Required elements include:

\begin{Shaded}
\begin{Highlighting}[]
\FunctionTok{title}\KeywordTok{:}\AttributeTok{ Title of the paper}
\FunctionTok{author}\KeywordTok{:}
\AttributeTok{  }\KeywordTok{{-}}\AttributeTok{ }\FunctionTok{name}\KeywordTok{:}\AttributeTok{ first and last name}
\FunctionTok{    affil}\KeywordTok{: }\CharTok{|}
\NormalTok{      One or more comma seperated numbers corresponding to affilitation}
\NormalTok{      and one or more  comma seperated symbols corresponding }
\NormalTok{      optional notes.}
\AttributeTok{    }\FunctionTok{orcid}\KeywordTok{:}\AttributeTok{ optional orcid number}
\FunctionTok{affiliation}\KeywordTok{:}\AttributeTok{  }
\AttributeTok{  }\KeywordTok{{-}}\AttributeTok{ }\FunctionTok{num}\KeywordTok{:}\AttributeTok{ 1,..., n for each affiliation}
\AttributeTok{    }\FunctionTok{address}\KeywordTok{:}\AttributeTok{ required}
\AttributeTok{    }\FunctionTok{email}\KeywordTok{:}\AttributeTok{ required}
\FunctionTok{authorcitation}\KeywordTok{: }\CharTok{|}
\NormalTok{  Lastname, F.}
\FunctionTok{correspondence}\KeywordTok{: }\CharTok{|}
\NormalTok{  email@email.com; Tel.: +XX{-}000{-}00{-}0000.}
\FunctionTok{journal}\KeywordTok{:}\AttributeTok{ notspecified}
\FunctionTok{type}\KeywordTok{:}\AttributeTok{ article}
\FunctionTok{status}\KeywordTok{:}\AttributeTok{ submit}
\end{Highlighting}
\end{Shaded}

Journal options are in Table \ref{tab:mdpinames}. The \texttt{status}
variable should generally not be changed by authors. The \texttt{type}
variable describes the type of of submission and defaults to
\texttt{article} but can be replaced with any of the ones in Table
\ref{tab:mdpitype}

\begin{table}

\caption{\label{tab:mdpitype}MDPI article types.}
\centering
\begin{tabular}[t]{lll}
\toprule
abstract & entry & retraction\\
addendum & expressionofconcern & review\\
article & extendedabstract & perspective\\
book & datadescriptor & protocol\\
bookreview & editorial & shortnote\\
\addlinespace
briefreport & essay & studyprotocol\\
casereport & erratum & systematicreview\\
comment & hypothesis & supfile\\
commentary & interestingimage & technicalnote\\
communication & obituary & viewpoint\\
\addlinespace
conferenceproceedings & opinion & guidelines\\
correction & projectreport & registeredreport\\
conferencereport & reply & tutorial\\
\bottomrule
\end{tabular}
\end{table}

\subsection{Journal Specific YAML
variables}\label{journal-specific-yaml-variables}

\begin{Shaded}
\begin{Highlighting}[]
\CommentTok{\# for journal Diversity,}
\CommentTok{\# add the Life Science Identifier using:}
\FunctionTok{lsid}\KeywordTok{:}\AttributeTok{ http://zoobank.org/urn:lsid:zoobank.org:act:nnnn}


\CommentTok{\# for journal Applied Sciences}
\CommentTok{\# add featured application}
\FunctionTok{featuredapplication}\KeywordTok{: }\CharTok{|}
\NormalTok{  Authors are encouraged to provide a concise }
\NormalTok{  description of the specific application or }
\NormalTok{  a potential application of the work. This }
\NormalTok{  section is not mandatory.}

\CommentTok{\# for the journal Data}
\CommentTok{\# add dataset doi and license}
\FunctionTok{dataset}\KeywordTok{:}\AttributeTok{ https://doi.org/10.1000/182}
\FunctionTok{datasetlicense}\KeywordTok{:}\AttributeTok{ CC{-}BY{-}4.0}

\CommentTok{\# for the journal Toxins}
\CommentTok{\# add key contributions}
\FunctionTok{keycontributions}\KeywordTok{: }\CharTok{|}
\NormalTok{  The breakthroughs or highlights of the manuscript. }
\NormalTok{  Authors can write one or two sentences to describe }
\NormalTok{  the most important part of the paper.}

\CommentTok{\# for the journal Encyclopedia}
\FunctionTok{encyclopediadef}\KeywordTok{: }\CharTok{|}
\NormalTok{  For entry manuscripts only: please provide a brief overview}
\NormalTok{  of the entry title instead of an abstract.}
\FunctionTok{entrylink}\KeywordTok{:}\AttributeTok{ The Link to this entry published on the encyclopedia platform.}

\CommentTok{\# for the journal Advances in Respiratory Medicine}
\CommentTok{\# add highlights}
\FunctionTok{addhighlights}\KeywordTok{: }\CharTok{|}
\NormalTok{  This is an obligatory section in “Advances in Respiratory Medicine”, }
\NormalTok{  whose goal is to increase the discoverability and readability of the}
\NormalTok{  article via search engines and other scholars. Highlights should not }
\NormalTok{  be a copy of the abstract, but a simple text allowing the reader to }
\NormalTok{  quickly and simplified find out what the article is about and what can }
\NormalTok{  be cited from it. Each of these parts should be devoted up to 2\textasciitilde{}bullet }
\NormalTok{  points.}
\end{Highlighting}
\end{Shaded}

\startlandscape

\begin{longtable}[t]{llllllll}
\caption{\label{tab:mdpinames}MDPI journal names.}\\
\toprule
atmosphere & coatings & engproc & hospitals & jsan & mti & powders & stresses\\
atoms & colloids & entomology & humanities & jtaer & muscles & preprints & surfaces\\
audiolres & colorants & entropy & humans & jvd & nanoenergyadv & proceedings & surgeries\\
automation & commodities & environments & hydrobiology & jzbg & nanomanufacturing & processes & suschem\\
axioms & compounds & environsciproc & hydrogen & kidneydial & nanomaterials & prosthesis & sustainability\\
\addlinespace
bacteria & computation & epidemiologia & hydrology & kinasesphosphatases & ncrna & proteomes & symmetry\\
batteries & computers & epigenomes & hygiene & knowledge & ndt & psf & synbio\\
bdcc & condensedmatter & est & idr & land & network & psych & systems\\
behavsci & conservation & fermentation & ijerph & languages & neuroglia & psychiatryint & targets\\
beverages & constrmater & fibers & ijfs & laws & neurolint & psychoactives & taxonomy\\
\addlinespace
biochem & cosmetics & fintech & ijgi & life & neurosci & publications & technologies\\
bioengineering & covid & fire & ijms & liquids & nitrogen & quantumrep & telecom\\
biologics & crops & fishes & ijns & literature & notspecified & quaternary & test\\
biology & cryptography & fluids & ijpb & livers & nri & qubs & textiles\\
biomass & crystals & foods & ijtm & logics & nursrep & radiation & thalassrep\\
\addlinespace
biomechanics & csmf & forecasting & ijtpp & logistics & nutraceuticals & reactions & thermo\\
biomed & ctn & forensicsci & ime & lubricants & nutrients & receptors & tomography\\
biomedicines & curroncol & forests & immuno & lymphatics & obesities & recycling & tourismhosp\\
biomedinformatics & cyber & foundations & informatics & machines & oceans & regeneration & toxics\\
biomimetics & dairy & fractalfract & information & macromol & ohbm & religions & toxins\\
\addlinespace
biomolecules & data & fuels & infrastructures & magnetism & onco & remotesensing & transplantology\\
biophysica & ddc & future & inorganics & magnetochemistry & oncopathology & reports & transportation\\
biosensors & dentistry & futureinternet & insects & make & optics & reprodmed & traumacare\\
biotech & dermato & futurepharmacol & instruments & marinedrugs & oral & resources & traumas\\
birds & dermatopathology & futurephys & inventions & materials & organics & rheumato & tropicalmed\\
\addlinespace
bloods & designs & futuretransp & iot & materproc & organoids & risks & universe\\
blsf & devices & galaxies & j & mathematics & osteology & robotics & urbansci\\
brainsci & diabetology & games & jal & mca & oxygen & ruminants & uro\\
breath & diagnostics & gases & jcdd & measurements & parasites & safety & vaccines\\
buildings & dietetics & gastroent & jcm & medicina & parasitologia & sci & vehicles\\
\addlinespace
businesses & digital & gastrointestdisord & jcp & medicines & particles & scipharm & venereology\\
cancers & disabilities & gels & jcs & medsci & pathogens & sclerosis & vetsci\\
carbon & diseases & genealogy & jcto & membranes & pathophysiology & seeds & vibration\\
cardiogenetics & diversity & genes & jdb & merits & pediatrrep & sensors & virtualworlds\\
catalysts & dna & geographies & jeta & metabolites & pharmaceuticals & separations & viruses\\
\addlinespace
cells & drones & geohazards & jfb & metals & pharmaceutics & sexes & vision\\
ceramics & dynamics & geomatics & jfmk & meteorology & pharmacoepidemiology & signals & waste\\
challenges & earth & geosciences & jimaging & methane & pharmacy & sinusitis & water\\
chemengineering & ebj & geotechnics & jintelligence & metrology & philosophies & skins & wem\\
chemistry & ecologies & geriatrics & jlpea & micro & photochem & smartcities & wevj\\
\addlinespace
chemosensors & econometrics & grasses & jmmp & microarrays & photonics & sna & wind\\
chemproc & economies & gucdd & jmp & microbiolres & phycology & societies & women\\
children & education & hazardousmatters & jmse & micromachines & physchem & socsci & world\\
chips & ejihpe & healthcare & jne & microorganisms & physics & software & youth\\
cimb & electricity & hearts & jnt & microplastics & physiologia & soilsystems & zoonoticdis\\
\addlinespace
civileng & electrochem & hemato & jof & minerals & plants & solar & \\
cleantechnol & electronicmat & hematolrep & joitmc & mining & plasma & solids & \\
climate & electronics & heritage & jor & modelling & platforms & spectroscj & \\
clinpract & encyclopedia & higheredu & journalmedia & molbank & pollutants & sports & \\
clockssleep & endocrines & highthroughput & jox & molecules & polymers & standards & \\
\addlinespace
cmd & energies & histories & jpm & mps & polysaccharides & stats & \\
coasts & eng & horticulturae & jrfm & msf & poultry & std & \\
\bottomrule
\end{longtable}
\finishlandscape

\section{Introduction}\label{introduction-1}

The introduction should briefly place the study in a broad context and
highlight why it is important. It should define the purpose of the work
and its significance. The current state of the research field should be
reviewed carefully and key publications cited. Please highlight
controversial and diverging hypotheses when necessary. Finally, briefly
mention the main aim of the work and highlight the principal
conclusions. As far as possible, please keep the introduction
comprehensible to scientists outside your particular field of research.
Citing a journal paper
\citep{bertrand-krajewski_distribution_1998, leutnant_stormwater_2016}.
And now citing a book reference \citet{gujer_systems_2008}. Some MDPI
journals use Chicago and others use APA, this template should choose the
correct citation format for you once you specify the journal in the YAML
header.

To use endnotes, change \texttt{endnotes:\ true} in the YAML header,
then use \texttt{\textbackslash{}endnote\{This\ is\ an\ endnote.\}}.

\section{Materials and Methods}\label{materials-and-methods-1}

Materials and Methods should be described with sufficient details to
allow others to replicate and build on published results. Please note
that publication of your manuscript implicates that you must make all
materials, data, computer code, and protocols associated with the
publication available to readers. Please disclose at the submission
stage any restrictions on the availability of materials or information.
New methods and protocols should be described in detail while
well-established methods can be briefly described and appropriately
cited.

Research manuscripts reporting large datasets that are deposited in a
publicly available database should specify where the data have been
deposited and provide the relevant accession numbers. If the accession
numbers have not yet been obtained at the time of submission, please
state that they will be provided during review. They must be provided
prior to publication.

Interventionary studies involving animals or humans, and other studies
require ethical approval must list the authority that provided approval
and the corresponding ethical approval code.

\section{Results}\label{results-1}

This section may be divided by subheadings. It should provide a concise
and precise description of the experimental results, their
interpretation as well as the experimental conclusions that can be
drawn.

\subsection{Subsection Heading Here}\label{subsection-heading-here}

Subsection text here.

\subsubsection{Subsubsection Heading
Here}\label{subsubsection-heading-here}

Bulleted lists look like this:

\begin{itemize}
\tightlist
\item
  First bullet
\item
  Second bullet
\item
  Third bullet
\end{itemize}

Numbered lists can be added as follows:

\begin{enumerate}
\def\labelenumi{\arabic{enumi}.}
\tightlist
\item
  First item
\item
  Second item
\item
  Third item
\end{enumerate}

The text continues here.

\subsection{Figures, Tables and
Schemes}\label{figures-tables-and-schemes}

All figures and tables should be cited in the main text as Figure
\ref{fig:fig1}, \ref{tab:tab1}, etc. To get cross-reference to figure
generated by R chunks include the
\texttt{\textbackslash{}\textbackslash{}label\{\}} tag in the
\texttt{fig.cap} attribute of the R chunk:

When making tables using \texttt{kable}, it is suggested to use the
\texttt{format="latex"} and \texttt{tabl.envir="table"} arguments to
ensure table numbering and compatibility with the mdpi document class.

\begin{table}[H]

\caption{\label{tab:tab1}This is a table caption. Tables should be placed in the 
             main text near to the first time they are~cited.}
\begin{tabular}[t]{lccc}
\toprule
  & mpg & cyl & disp\\
\midrule
Mazda RX4 & 21.0 & 6 & 160\\
Mazda RX4 Wag & 21.0 & 6 & 160\\
Datsun 710 & 22.8 & 4 & 108\\
Hornet 4 Drive & 21.4 & 6 & 258\\
Hornet Sportabout & 18.7 & 8 & 360\\
\bottomrule
\end{tabular}
\end{table}

For a very wide table, landscape layouts are allowed.

\begin{table}[H]

\caption{\label{tab:tab2}This is a very wide table}
\begin{tabular}[t]{cccc}
\toprule
Title.1 & Title.2 & Title.3 & Title.4\\
\midrule
Entry 1 & Data & Data & This cell has some longer content that runs over
                               two lines\\
Entry 2 & Data & Data & Data\\
\bottomrule
\end{tabular}
\end{table}

\finishlandscape

\subsection{Formatting of Mathematical
Components}\label{formatting-of-mathematical-components}

This is an example of an equation:

\[
a = 1.
\]

If you want numbered equations use Latex and wrap in the equation
environment:

\begin{equation}
a = 1,
\end{equation}

the text following an equation need not be a new paragraph. Please
punctuate equations as regular text.

This is the example 2 of equation:

\begin{adjustwidth}{-\extralength}{0cm}
\begin{equation}
a = b + c + d + e + f + g + h + i + j + k + l + m + n + o + p + q + r + s + t + 
u + v + w + x + y + z
\end{equation}
\end{adjustwidth}

Theorem-type environments (including propositions, lemmas, corollaries
etc.) can be formatted as follows:

Example of a theorem:

\begin{Theorem}
Example text of a theorem

\end{Theorem}

The text continues here. Proofs must be formatted as follows:

Example of a proof:

\begin{proof}[Proof of Theorem1]
Text of the proof. Note that the phrase ``of Theorem 1'\,' is optional
if it is clear which theorem is being referred to.

\end{proof}

The text continues here.

\section{Discussion}\label{discussion-1}

Authors should discuss the results and how they can be interpreted in
perspective of previous studies and of the working hypotheses. The
findings and their implications should be discussed in the broadest
context possible. Future research directions may also be highlighted.

\section{Conclusion}\label{conclusion}

This section is not mandatory, but can be added to the manuscript if the
discussion is unusually long or complex.

\section{Patents}\label{patents}

This section is not mandatory, but may be added if there are patents
resulting from the work reported in this manuscript.

%%%%%%%%%%%%%%%%%%%%%%%%%%%%%%%%%%%%%%%%%%

\vspace{6pt}

%%%%%%%%%%%%%%%%%%%%%%%%%%%%%%%%%%%%%%%%%%
%% optional

% Only for the journal Methods and Protocols:
% If you wish to submit a video article, please do so with any other supplementary material.
% \supplementary{The following supporting information can be downloaded at: \linksupplementary{s1}, Figure S1: title; Table S1: title; Video S1: title. A supporting video article is available at doi: link.}

%%%%%%%%%%%%%%%%%%%%%%%%%%%%%%%%%%%%%%%%%%
\authorcontributions{For research articles with several authors, a short
paragraph specifying their individual contributions must be provided.
The following statements should be used ``X.X. and Y.Y. conceive and
designed the experiments; X.X. performed the experiments; X.X. and Y.Y.
analyzed the data; W.W. contributed reagents/materials/analysis tools;
Y.Y. wrote the paper.'\,' Authorship must be limited to those who have
contributed substantially to the work reported.}

\funding{Please add:
\texttt{This\ research\ received\ no\ external\ funding\textquotesingle{}\textquotesingle{}\ or}This
research was funded by NAME OF FUNDER grant number XXX.'\,' and and
``The APC was funded by XXX'\,'. Check carefully that the details given
are accurate and use the standard spelling of funding agency names at
\url{https://search.crossref.org/funding}, any errors may affect your
future funding.}

\institutionalreview{Ethical review and approval were waived for this
study due to the nature of the research not involving human subjects or
animals directly.}

\informedconsent{Not applicable.}

\dataavailability{We encourage all authors of articles published in MDPI
journals to share their research data. In this section, please provide
details regarding where data supporting reported results can be found,
including links to publicly archived datasets analyzed or generated
during the study. Where no new data were created, or where data is
unavailable due to privacy or ethical restrictions, a statement is still
required. Suggested Data Availability Statements are available in
section ``MDPI Research Data Policies'' at
\url{https://www.mdpi.com/ethics}.}

\acknowledgments{All sources of funding of the study should be
disclosed. Please clearly indicate grants that you have received in
support of your research work. Clearly state if you received funds for
covering the costs to publish in open access.}

\conflictsofinterest{The authors declare no conflict of interest.}

%%%%%%%%%%%%%%%%%%%%%%%%%%%%%%%%%%%%%%%%%%
%% Optional

%% Only for journal Encyclopedia


%%%%%%%%%%%%%%%%%%%%%%%%%%%%%%%%%%%%%%%%%%
%% Optional
%%%%%%%%%%%%%%%%%%%%%%%%%%%%%%%%%%%%%%%%%%
\begin{adjustwidth}{-\extralength}{0cm}

%\printendnotes[custom] % Un-comment to print a list of endnotes


\reftitle{References}
\bibliography{mybibfile.bib}

% If authors have biography, please use the format below
%\section*{Short Biography of Authors}
%\bio
%{\raisebox{-0.35cm}{\includegraphics[width=3.5cm,height=5.3cm,clip,keepaspectratio]{Definitions/author1.pdf}}}
%{\textbf{Firstname Lastname} Biography of first author}
%
%\bio
%{\raisebox{-0.35cm}{\includegraphics[width=3.5cm,height=5.3cm,clip,keepaspectratio]{Definitions/author2.jpg}}}
%{\textbf{Firstname Lastname} Biography of second author}

%%%%%%%%%%%%%%%%%%%%%%%%%%%%%%%%%%%%%%%%%%
%% for journal Sci
%\reviewreports{\\
%Reviewer 1 comments and authors’ response\\
%Reviewer 2 comments and authors’ response\\
%Reviewer 3 comments and authors’ response
%}
%%%%%%%%%%%%%%%%%%%%%%%%%%%%%%%%%%%%%%%%%%
\PublishersNote{}
\end{adjustwidth}


\end{document}
